\documentclass{beamer}
\usepackage{sansmathaccent}
\usepackage{amsmath}
\usepackage{braket}
\usepackage{mathtools}
\usepackage[thinc]{esdiff}
\usepackage{tikz}
\usetikzlibrary{quantikz}
\pdfmapfile{+sansmathaccent.map}
\usetheme{Boadilla}
\title{Quantum Computing}
\subtitle{Teach Me X}
\author{Giacomo Fenzi}
\institute{University of St. Andrews}
\date{\today}
\begin{document}
\begin{frame}
    \titlepage
\end{frame}
\begin{frame}
    \frametitle{Outline}
    \tableofcontents
\end{frame}
\section{Mathematics}
\subsection{Vector Spaces}
\begin{frame}
    \frametitle{Vectors}
    A vector is an array of $n$ numbers. \\
    Let $S$ be a field. We write $S^n$ as the set of all vectors with $n$ entries in $S$.
    \begin{example}
        For example, $\mathbb{R}^n$, $\mathbb{C}^n$  are the sets of vectors of $n$, respectively, real and complex numbers. \\
        Here are a couple of vectors:
        \begin{align}
            \begin{pmatrix}
                1 \\
                2 \\
                3
            \end{pmatrix} & \in \mathbb{R}^3,
            \begin{pmatrix}
                1 + i \\
                1 - i
            \end{pmatrix} \in \mathbb{C}^2,
        \end{align}
    \end{example}
\end{frame}
\begin{frame}
    \frametitle{Notation}
    \begin{itemize}
        \item In school often vectors are denoted as $\boldsymbol{v}$ or $\vec{v}$.
        \item In Quantum mechanics we use Dirac notation: $\ket{v}$
        \item This just means the vector named $v$.
    \end{itemize}
\end{frame}
\begin{frame}
    \frametitle{Operation}
    Vectors are characterized by two operations on their values:
    \begin{itemize}
        \item Addition: given two vectors $\ket{a}, \ket{b}$, then $\ket{a} + \ket{b}$ is also a vector, defined by summing pointwise the components of the two.
        \item Scalar multiplication: given a vector $\ket{a}$ and a scalar $\alpha$, then $\alpha \ket{a}$ is a vector, defined by multiplying each component of $\ket{a}$ by $\alpha$.
    \end{itemize}
    \begin{example}
        Let:
        \begin{align}
            \ket{a} & = \begin{pmatrix}
                1 \\
                2
            \end{pmatrix},
            \ket{b} = \begin{pmatrix}
                3 \\
                4
            \end{pmatrix}
        \end{align}
        Then:
        \begin{align}
            2\ket{a} + \ket{b} & = \begin{pmatrix}
                5 \\
                8
            \end{pmatrix},
        \end{align}
    \end{example}
\end{frame}
\begin{frame}
    \frametitle{Vector Spaces and Basis}
    A set like $\mathbb{R}^n, \mathbb{C}^n$ is called\footnote{Pedantically, it's the other way around, we first define vectors spaces and prove the two are one} a \textit{vector space}
    In particular, each vector space $V$ contains some set of vectors called a basis for the space, which we denote $\ket{0}, \ket{1}, \dots, \ket{n - 1}$ such that:
    \begin{itemize}
        \item For every $\ket{v} \in V$, there exist $\alpha_i$ such that: \[ \ket{v} = \sum_i \alpha_i \ket{i}  \]
        \item \[ \sum_i \alpha_i \ket{i} = 0 \] iff $\alpha_i = 0$.
    \end{itemize}
\end{frame}
\begin{frame}
    \frametitle{Basis Examples}
    \begin{example}
        Consider the vector space $\mathbb{R}^2$. This is the so called standard basis:
        \begin{align}
            \ket{0} & = \begin{pmatrix}
                1 \\
            \end{pmatrix}, \ket{1} = \begin{pmatrix}
                0 \\
                1
            \end{pmatrix}
        \end{align}
        Also, the following is an interesting basis we will encounter:
        \begin{align}
            \ket{+} & = \frac{1}{\sqrt{2}} \begin{pmatrix}
                1 \\
                1
            \end{pmatrix}, \ket{-} = \frac{1}{\sqrt{2}} \begin{pmatrix}
                1 \\
                -1
            \end{pmatrix}
        \end{align}
        The proof is quite simple, I suggest first showing the standard basis is one, and then note that:
        \[ \ket{0} = \frac{\ket{+} + \ket{-}}{\sqrt{2}}, \ket{1} = \frac{\ket{+} - \ket{-}}{\sqrt{2}} \]
    \end{example}
\end{frame}
\subsection{Linear operators}
\begin{frame}
    \frametitle{Linear Transformation}
    A linear transformation between two vectors spaces is a function $T: V \to W$ such that, for $\ket{a}, \ket{b} \in V$, and scalar $\alpha$
    \begin{itemize}
        \item $T(\ket{a} + \ket{b}) = T\ket{a} + T\ket{b}$
        \item $T(\alpha\ket{a})  = \alpha T\ket{a}$
    \end{itemize}
    Since every vector in $V$ can be written as a sum of scaled basis vectors it follows:
    \[ T\ket{v} = \sum_i \alpha_i T\ket{i}  \]
    And a linear transformation can be uniquely determined by its action on the bases. \\
    Also, if $V = W$, we call $T: V \to V$ a \textit{linear operator} on $V$.
\end{frame}
\begin{frame}
    \frametitle{Matrix}
    Encoding this operation, we can write a linear transformation as a matrix (and viceversa).
    \begin{example}
        Consider the following linear operator $T$ on $\mathbb{C}^2$:
        \begin{align}
            T\ket{0} & = \frac{1}{\sqrt{2}}(\ket{0} + \ket{1}) \\
            T\ket{1} & = \frac{1}{\sqrt{2}}(\ket{0} - \ket{1})
        \end{align}
        Then the corresponding matrix is the useful \textit{Hadamard Gate}
        \begin{equation}
            \label{Hadamard}
            H = \frac{1}{\sqrt{2}}
            \begin{bmatrix}
                1 & 1  \\
                1 & -1
            \end{bmatrix}
        \end{equation}
    \end{example}
\end{frame}
\begin{frame}
    \frametitle{Pauli Matrices}
    \begin{example}
        The following matrices are of particular importance, and $X, Y, Z$ are called the Pauli Matrices.
        \begin{align}
            \label{Pauli}
            I & =
            \begin{bmatrix}
                1 & 0 \\
                0 & 1
            \end{bmatrix}, \qquad
            X = \begin{bmatrix}
                0 & 1 \\
                1 & 0
            \end{bmatrix} \\
            Y & =
            \begin{bmatrix}
                0 & -i \\
                i & 0
            \end{bmatrix}, \qquad
            Z = \begin{bmatrix}
                1 & 0  \\
                0 & -1
            \end{bmatrix}
        \end{align}
    \end{example}
\end{frame}
\subsection{Hilbert Spaces}
\begin{frame}
    \frametitle{Inner Product}
    The inner product is a generalization of the dot product to complex vectors in $\mathbb{C}^n$.
    We write $\braket{v|w}$ for the inner product of $\ket{v}, \ket{w}$.
    It has the properties that:
    \begin{itemize}
        \item $(\braket{v|w})^* = \braket{w|v}$, where $\cdot^*$ is the complex conjugate.
        \item $\braket{v|v} \geq 0$. We define also $||\ket{v}|| \equiv \sqrt{\braket{v|v}}$
        \item If $\ket{w} = \sum_i \alpha_i \ket{w_i}$, then $\braket{v|w} = \sum_i \alpha_i \braket{v | w_i}$
    \end{itemize}
    We define an Hilbert space to be a vector space with such inner product.
    In $\mathbb{C}^n$ we define that if:
    \begin{align}
        \ket{v} & = \begin{pmatrix}
            v_1    \\
            \vdots \\
            v_n
        \end{pmatrix},
        \ket{w} = \begin{pmatrix}
            w_1    \\
            \vdots \\
            w_n
        \end{pmatrix} \implies
        \braket{v|w} = \sum_i v_i^* w_n = (v_1^* \dots v_n^*)\begin{pmatrix}
            w_1    \\
            \vdots \\
            w_n
        \end{pmatrix}
    \end{align}
    Also, if $|| \ket{\psi} || = 1$, we say that $\ket{\psi}$ is a unit vector.
\end{frame}
\begin{frame}
    \frametitle{Inner Products}
    \begin{example}
        Consider the following (all in $\mathbb{C}^2$):
        \begin{align}
            \ket{a} & = \begin{pmatrix}
                1 + i \\
                2
            \end{pmatrix},
            \ket{b} = \begin{pmatrix}
                2 \\
                3
            \end{pmatrix}
        \end{align}
        And $\ket{0}, \ket{1}, \ket{+}, \ket{-}$ as defined before.
        Then:
        \begin{equation} \braket{0|1} = \braket{+| -} = 0 \end{equation}
        \begin{equation} \braket{0|0} = \braket{1|1} =  \braket{+| +} = \braket{-| -} = 1 \end{equation}
        \begin{equation}
            \braket{a|b} = 6 - 2i, \braket{b|b} = 13, \braket{a|0} = 1-i
        \end{equation}
    \end{example}

\end{frame}
\begin{frame}
    \frametitle{Orthonormality}
    You might have noticed before that our standard basis (and in fact also $\ket{+}, \ket{-}$) of $\mathbb{C}^2$ satisfies the following property:
    \begin{equation}
        \braket{i|j} = \delta_{ij}
    \end{equation}
    Where $\delta_{ij}$ is the Kronecker delta, defined as:
    \begin{equation}
        \delta_{ij} =
        \begin{cases}
            1, & \text{if } i=j,     \\
            0, & \text{if } i\neq j.
        \end{cases}
    \end{equation}
    Any basis, satisfying this condition is called \textit{orthonormal}.
    A result in linear algebra is that any basis can be converted to an orthonormal one thanks to the Gram Schmidt process, which simplifies a lot of proofs.
\end{frame}
\subsection{Eigenstuff}
\begin{frame}
    \frametitle{Eigenvalues}
    In the whole of linear algebra one equation is of particular interest:
    \begin{equation}
        A\ket{v} = \lambda \ket{v}
    \end{equation}
    If the above hold, then $\lambda$ is called an \textit{eigenvalue} of $A$, with $\ket{v}$ its corresponding \textit{eigenvector}.
    We are particularly interested in \textit{spectral decompositions}, i.e. operators for which:
    \begin{equation}
        A = \sum_\lambda \lambda \ket{\lambda}\bra{\lambda}.
    \end{equation}
    Where $\lambda$ is the eigenvalue with eigenvector $\ket{\lambda}$, and $\ket{v}\bra{w}$ is defined to
    be the linear operator such that $\ket{v}\bra{w}(\ket{z}) \equiv \ket{v}\braket{w|z}$

\end{frame}
\begin{frame}
    \frametitle{Eigenvalues}
    \begin{example}
        Consider $H$ as in \ref{Hadamard}. Then $H$ has eigenvalues $\pm 1$, with (non normalized) eigenvectors $(1 \pm \sqrt{2})\ket{0} + \ket{1}$.
        The Pauli Matrices defined in \ref{Pauli} also have eigenvalues $\pm 1$. The eigenvectors are:
        \begin{align}
            \label{decomposition}
            \ket{\lambda_{x+}} = \ket{+} & = \frac{1}{\sqrt{2}} \begin{pmatrix}
                1 \\
                1
            \end{pmatrix},
            \ket{\lambda_{x-}} = \ket{-} = \frac{1}{\sqrt{2}} \begin{pmatrix}
                1 \\
                -1
            \end{pmatrix}
        \end{align}
        \begin{align}
            \ket{\lambda_{y+}} & = \frac{1}{\sqrt{2}} \begin{pmatrix}
                1 \\
                i
            \end{pmatrix},
            \ket{\lambda_{y-}} = \frac{1}{\sqrt{2}} \begin{pmatrix}
                i \\
                1
            \end{pmatrix}
        \end{align}
        \begin{align}
            \ket{\lambda_{z+}} & = \ket{0} = \begin{pmatrix}
                1 \\
            \end{pmatrix},
            \ket{\lambda_{z-}} = \ket{1} = \begin{pmatrix}
                0 \\
                1
            \end{pmatrix}
        \end{align}
    \end{example}
\end{frame}
\subsection{Adjoint and Tensors}
\begin{frame}
    \frametitle{Adjoint}
    For a linear operator $A$, we define $A^\dagger \equiv (A^T)^*$, where $\cdot^T$ is the transpose operation.
    It has the defining property that $\braket{v|A|w} = \braket{v'|w}$ where $\ket{v'} = A^\dagger \ket{v}$.
    Some matrices behave particularly well with respect with their adjoint, and so we denote then appropriately:
    \begin{itemize}
        \item $H = H^\dagger$, then $H$ is Hermitian
        \item $UU^\dagger = I$, then $U$ is Unitary
        \item $N N^\dagger = N^\dagger N$, then $N$ is Normal
    \end{itemize}
    It is quite easy to show that an Hermitian or Unitary operator is automatically normal, and
    the spectral theorem shows that any Normal operator has a spectral decompositions, which comes in quite handy.
\end{frame}
\begin{frame}
    \frametitle{Tensor Product}
    Finally, we want a way to put together Hilbert spaces. For Hilbert spaces $V, W$ we define the tensor product $V \otimes W$.
    If $\ket{v_i}, \ket{w_i}$ is a basis for respectively $V, W$, then $\ket{v_i} \otimes \ket{w_i}$ is a basis for the product.
    Also the following hold:
    \begin{itemize}
        \item $\alpha (\ket{v}\otimes\ket{w}) = (\alpha \ket{v})\otimes\ket{w} = \ket{v}\otimes (\alpha \ket{w})$
        \item $(\ket{v} + \ket{v'}) \otimes\ket{w} = \ket{v} \otimes \ket{w} + \ket{v'} \otimes \ket{w}$
        \item $\ket{v} \otimes (\ket{w} + \ket{w'}) = \ket{v} \otimes \ket{w} + \ket{v} \otimes \ket{w'}$
    \end{itemize}
    Also, for linear operators $A: V \to V$, $B: W \to W$ we define $(A \otimes B) (\ket{v} \otimes \ket{w}) \equiv A \ket{v} \otimes B \ket{w}$
    Finally, the inner product of $\ket{a} = \sum_i \alpha_i \ket{v_i} \otimes \ket{w_i}$ and $\ket{b} = \sum_j \beta_j \ket{v_j'} \otimes \ket{w_j'}$ is quite\footnote{lol} naturally:
    \begin{equation}
        \braket{a|b} \equiv \sum_{ij} \alpha_i^* \beta_j \braket{v_i|v_j'} \braket{w_i|w_j'}
    \end{equation}
    Or, more simply: $(\bra{v} \otimes \bra{w})(\ket{v'} \otimes \ket{w'}) = \braket{v|v'} \braket{w|w'}$
\end{frame}
\begin{frame}
    \frametitle{Notes on notation}
    It is common to abbreviate $\ket{v} \otimes \ket{w}$ as $\ket{v}\ket{w}$ or $\ket{v, w}$ or even,
    sometimes (especially in qubit systems) as $\ket{v_L w_R}$ or $\ket{v w}$. \\
    Also we have the operation $\cdot^{\otimes n}$ is the operation of tensoring the object with itself $n$ times.
    \begin{example}
        Consider  $\mathbb{C}^2 \otimes \mathbb{C}^2 = (\mathbb{C}^2)^{\otimes 2}$.
        This vector space has then basis $\ket{0}\ket{0}, \ket{0}\ket{1}, \ket{1}\ket{0}, \ket{1}\ket{1}$.
        Then $\ket{0}^{\otimes 2} = \ket{0} \otimes \ket{0} = \ket{00}$
        Note that here I am switching notation just for explanation, I'll try to be consistent afterwards.
    \end{example}



\end{frame}
\begin{frame}
    \frametitle{Kronecker product}
    In order to make this more concrete, we define the Kronecker product for tensoring two matrices (vec) together.
    Let $A$ be a $m \times n$ matrix, and $B$ be a $p \times q$ matrix. Then
    \begin{align}
        A \otimes B & = \begin{pmatrix}
            A_{11} B \dots A_{1n} B \\
            \vdots \ddots \vdots    \\
            A_{m1} B \dots A_{mn} B
        \end{pmatrix}
    \end{align}
    is a $mp \times nq$ matrix, and is called the Kronecker product of $A, B$.

\end{frame}
\begin{frame}
    \frametitle{Tensor Product}
    \begin{example}
        Consider now the the tensor product $T = (\mathbb{C}^2)^{\otimes 2}$.
        The vector $(\ket{0} + 2 \ket{1}) \otimes (2 \ket{0} + 3 \ket{1})$ in $T$ is equal to $ 2 \ket{00} + 3 \ket{01} + 4\ket{10} + 6\ket{11}$
        This is also apparent from the Kronecker representation:
        \begin{equation}
            \begin{pmatrix}
                1 \\
                2
            \end{pmatrix} \otimes
            \begin{pmatrix}
                2 \\
                3
            \end{pmatrix}
            =
            \begin{pmatrix}
                1 \times 2 \\
                1 \times 3 \\
                2 \times 2 \\
                2 \times 3
            \end{pmatrix}
            = \begin{pmatrix}
                2 \\
                3 \\
                4 \\
                6
            \end{pmatrix}
        \end{equation}
        You can also check that the product of Pauli matrices $X, Y$ is:
        \begin{equation}
            X \otimes Y =
            \begin{pmatrix}
                0 & 0  & 0 & -i \\
                0 & 0  & i & 0  \\
                0 & -i & 0 & 0  \\
                i & 0  & 0 & 0
            \end{pmatrix}
        \end{equation}
    \end{example}
\end{frame}
\section{Physics}
\subsection{Postulates}
\begin{frame}
    \frametitle{State space}
    Quantum Mechanics is a framework for understanding quantum systems.
    It has some fundamental postulates that it uses as axioms and are experimentally validated.
    \begin{definition}
        \textbf{Postulate 1}: Every closed quantum system has an associated Hilbert space.
        This space is known as the \textit{state space} of the system. The system is completely described by its
        \textit{state vector}, which is simply a unit vector in the state space.
    \end{definition}
    A general state vector is of the form:
    \begin{equation}
        \ket{\psi} = \sum_i \alpha_i \ket{\psi_i}
    \end{equation}
    We say that $\ket{\psi}$ is in a superposition of states, with state $\ket{\psi_i}$ having amplitude $\alpha_i$.
\end{frame}
\begin{frame}
    \frametitle{State space}
    \begin{example}
        Let us consider the simplest quantum system: a \textit{qubit}.
        This is simply a quantum system with an associated two dimensional state space.
        Letting $\ket{0}, \ket{1}$ be a orthonormal basis for the state space, a general state vector
        is described as:
        \begin{equation}
            \ket{\psi} = a \ket{0} + b \ket{1}
        \end{equation}
        And, since $\ket{\psi}$ is a unit, $\braket{\psi|\psi} = |a|^2 + |b|^2 = 1$
        For example, the following is a valid state for a qubit:
        \begin{equation}
            \frac{\ket{0} - \ket{1}}{\sqrt{2}}
        \end{equation}
        Where state $\ket{0}$ has amplitude $\frac{1}{\sqrt{2}}$ and state $\ket{1}$ has amplitude $\frac{-1}{\sqrt{2}}$
    \end{example}
\end{frame}
\begin{frame}
    \frametitle{System Evolution}
    \begin{definition}
        \textbf{Postulate 2}: The evolution of a closed Quantum System is described by a
        Unitary transformation. So if the state vector at time $t_1$ is $\ket{\psi}$,
        and at time $t_2$ it becomes $\ket{\psi'}$, then there exist a Unitary operator $U$,
        which depends only on $t_1, t_2$ such that:
        \begin{equation}
            \ket{\psi'} = U \ket{\psi}
        \end{equation}

    \end{definition}
    Some unitary transformation we might consider are the Pauli matrices that we introduced before.
    For example $X$ acts as a gate that flips the single qubit $a\ket{0} + b\ket{1} \xrightarrow{X} b\ket{0} + a\ket{1}$
\end{frame}
\begin{frame}
    \frametitle{System Evolution}
    Postulate 2 before deals only with a discrete step in time. It turns out we can refine this to deal
    with continuos time evolution, and doing so results in the famous Schrödinger equation.
    \begin{definition}
        \textbf{Postulate 2'}:  The evolution of a closed Quantum System is described by \textit{Schrödinger equation}:
        \begin{equation}
            \label{shroedinger}
            i \hbar \diff{\ket{\psi}}{t} = H \ket{\psi}
        \end{equation}
        Where $\hbar$ is the reduced Planck Constant, and $H$ is an Hermitian operator which needs to
        be figured out for the system in question.
    \end{definition}
    Sometime it is useful that, if $H$ is independent from time, and we know the state $\ket{\psi_0}$ at $t = 0$
    \begin{equation}
        \ket{\psi(t)} = \exp\left(\frac{-itH}{\hbar}\right)\ket{\psi_0}
    \end{equation}
\end{frame}
\begin{frame}
    \frametitle{Energy states}
    In particular, since $H$ is Hermitian in \ref{shroedinger}, it has spectral decomposition:
    \begin{equation}
        H = \sum_E E \ket{E}\bra{E}
    \end{equation}
    where all the eigenvalues $E$ are real\footnote{This is a property of Hermitian operator}.
    The states $\ket{E}$ are known as the \textit{energy eigenstates} (or sometimes stationary), and the $E$ is the corresponding
    energy of the eigenstate. The lowest energy and the corresponding eigenstate are known as the \textit{ground state energy} and the \textit{ground state}.
\end{frame}
\begin{frame}
    \frametitle{System Evolution}
    \begin{example}
        Consider $H = \hbar \omega Z$, for some $\omega > 0$. Then, using \ref{decomposition}, $H = \hbar \omega (\ket{+}\bra{+} - \ket{-}\bra{-})$
        and as such the energy eigenstate are $\ket{+}, \ket{-}$, with energies respectively $\hbar \omega, -\hbar\omega$.
        So the ground state is $\ket{-}$, with energy $-\hbar\omega$.
        Also, if $\ket{\psi_0} = \ket{0}$, then using the solution before we have:
        \begin{align}
            \ket{\psi(t)} & = \exp(-i \omega t Z)\ket{0}                                                       \\
                          & = (e^{-i \omega t} \ket{+}\bra{+} + e^{i \omega t} \ket{-}\bra{-})\ket{0}          \\
                          & = \frac{1}{\sqrt{2}} \left(e^{-i \omega t} \ket{+} + e^{i \omega t} \ket{-}\right)
        \end{align}
    \end{example}
\end{frame}
\begin{frame}
    \frametitle{Measurement}
    Now, we aim to find a way to measure the state of a system.
    \begin{definition}
        \textbf{Postulate 3}: Quantum measurement is defined by a collection of operators
        ${M_m}$ called \textit{measurement operators}. In particular, given that the system is
        in state $\ket{\psi}$, the probability of obtaining result $m$ is:
        \begin{equation}
            p(m) = \braket{\psi|M_m^\dagger M_m|\psi}
        \end{equation}
        And the state after that measurement $m$ changes to:
        \begin{equation}
            \ket{\psi'} = \frac{M_m \ket{\psi}}{\sqrt{p(m)}}
        \end{equation}
    \end{definition}
    It is important to also have that $\sum_m p(m) = 1$ for every $\ket{\psi}$.
    Equivalently, the operators need to satisfy $\sum_m M_m^\dagger M_m = I$
\end{frame}
\begin{frame}
    \frametitle{Measurement}
    \begin{example}
        Let us consider the usual qubit system, and measurement operators:
        \begin{align}
            M_0 & = \ket{0}\bra{0} \\
            M_1 & = \ket{1}\bra{1}
        \end{align}
        It is easy to verify completeness is satisfied. Now, consider the general state
        $\ket{\psi} = a\ket{0} + b\ket{1}$. Then
        \begin{align}
            p(0) & = |a|^2 \text{ and } \ket{\psi} \to \frac{a}{|a|} \ket{0} \\
            p(1) & = |b|^2 \text{ and } \ket{\psi} \to \frac{b}{|b|} \ket{1}
        \end{align}
        Taking a concrete example: the state $\ket{+}$ when measured collapses
        to $\ket{0}$ precisely half of the times.
        Also note that measuring it with operators $M_{\pm} = \ket{\pm}\bra{\pm}$
        yields the result $\ket{+}$ with probability 1.
    \end{example}
\end{frame}
\begin{frame}
    \frametitle{Measurement}
    An alternative way to describe measurement operators, is by using projective measurements.
    The operators are described by a single Hermitian operator, called an \textit{observable}:
    \begin{equation}
        M = \sum_m m P_m
    \end{equation}
    where $P_m$ is the projector\footnote{An operator such that $P^2 = P$ } onto the eigenspace of $M$ with eigenvalue $m$.
    Using those, we have that, if we are on state $\ket{\psi}$:
    \begin{align}
        p(m)        & = \braket{\psi|P_m|\psi}             \\
        \ket{\psi'} & = \frac{P_m \ket{\psi}}{\sqrt{p(m)}}
    \end{align}
    These are especially useful since $\mathbb{E}(M) = \braket{\psi|M|\psi}$.
    We also write $\braket{M} \equiv \mathbb{E}(M)$, and $(\Delta(M))^2 = \mathrm{Var}(M) = \braket{M^2} - \braket{M}^2$
\end{frame}
\begin{frame}
    \frametitle{Measurement}
    \begin{example}
        Consider for example the observable $Z$ as defined in \ref{Pauli}.
        Then we know from \ref{decomposition} that:
        \begin{equation}
            Z = \ket{0}\bra{0} - \ket{1}\bra{1}
        \end{equation}
        So measuring our usual single qubit state $\ket{+}$ yields $+1$ or $-1$ with
        probability $\frac{1}{2}$. In particular, the expected value and standard deviation of the observable
        are
        \begin{align}
            \braket{Z} & = \braket{+|Z|+} = \braket{+| -} = 0                         \\
            \Delta(Z)  & = \sqrt{\braket{Z^2} - \braket{Z}} = \sqrt{\braket{+|+}} = 1
        \end{align}
        Heisenberg uncertainty principle can be written in this language as:
        \begin{equation}
            \Delta(A)\Delta(B) \geq \frac{|\braket{\psi|[A, B]|\psi}|}{2}
        \end{equation}
    \end{example}
\end{frame}
\begin{frame}
    \frametitle{Notes on Phase}
    Often you will hear the term of phase, and of two states being not distinguishable
    up to a global phase. What this means is that state $\ket{\psi'} = e^{i\theta}\ket{\psi}$
    for some real $\theta$. These are called undistinguishable as, for any measurement
    operator $M_m$ we have that:
    \begin{align}
        p_{\psi'}(m) & = \braket{\psi'|M_m^\dagger M_m|\psi'}                        \\
                     & = \braket{\psi|e^{-i\theta}M_m^\dagger M_m e^{i\theta} |\psi} \\
                     & = \braket{\psi|M_m^\dagger M_m |\psi} = p_{\psi}(m)
    \end{align}
\end{frame}
\begin{frame}
    \frametitle{Composition}
    Our final postulate, involves putting together Quantum Systems.
    It should come as no surprise that it involves the Tensor Product:
    \begin{definition}
        \textbf{Postulate 4}: The state space of a composite physical system is the tensor
        product of the state spaces of the individual systems. Furthermore,
        if there are $n$ systems, and system $i$ is in state $\ket{\psi_i}$,
        then the state of the composite system is:
        \begin{equation}
            \bigotimes_{i=1}^n \ket{\psi_i} = \ket{\psi_1} \otimes\dots \otimes \ket{\psi_n}
        \end{equation}
    \end{definition}
\end{frame}
\begin{frame}
    \frametitle{Composition}
    \begin{example}
        Consider two single qubit system, in states $\ket{\psi_1} = \ket{+}$, and $\ket{\psi_2} =\ket{-}$.
        Recall that each of the systems should have a probability $\frac{1}{2}$ to be measured as $0, 1$.
        So we expect that the tensor product of the two should obey the probability composition.
        \begin{align}
            \ket{\psi_1} \otimes \ket{\psi_2} & = \ket{+}\ket{-}                                                     \\
                                              & = \frac{1}{2}\left(\ket{00} + \ket{01} - \ket{10} - \ket{11} \right)
        \end{align}
        Using the idea that measurement probability is the square of the amplitude
        \footnote{You can prove using measurement operator $M_\alpha = \ket{\alpha}\bra{\alpha}$ for $\alpha \in \{0, 1\}^2$}
        we see that each of the states can be measured with probability $\frac{1}{4}$ as expected.
    \end{example}
\end{frame}
\begin{frame}
    \frametitle{Entanglement}
    One of the most important yet obscure term in quantum mechanics is the term entanglement.
    Consider two quantum systems, with states spaces $V, W$. Then the composite space is the space: $V \otimes W$.
    We say that the system is in an entangled state $\ket{\beta}$ if there are no $\ket{\psi} \in V$, $\ket{\psi'} \in W$
    such that $\ket{\beta} = \ket{\psi} \otimes \ket{\psi'}$.
    Actually, it turns out most of the computational power in Quantum computing comes from entangled states.
    To see this, note that for a $n$ Qubit system, non entangled states can be written as:
    \begin{equation}
        \bigotimes_{i =1}^n (a_i \ket{0} + b_i \ket{1}) = (a_1 \ket{0} + b_1 \ket{1}) \otimes \dots \otimes (a_n \ket{0} + b_n \ket{1})
    \end{equation}
    And so they are described by $n$ independent amplitudes. Instead a general state of the systems is of the form:
    \begin{equation}
        \sum_{\alpha \in \{0,1\}^n} a_\alpha \ket{\alpha} =  a_1 \ket{00\dots0} + a_2 \ket{00\dots1} + \dots + a_n \ket{11\dots1}
    \end{equation}
    Which has $2^n$ independent amplitudes.
\end{frame}
\begin{frame}
    \frametitle{Entanglement}
    \begin{example}
        Consider a two qubit system. Then some simple algebra shows that the following are entangled.
        \begin{align}
            \label{Bell states}
            \ket{\beta_{00}} & = \frac{\ket{00} + \ket{11}}{\sqrt{2}} \\
            \ket{\beta_{01}} & = \frac{\ket{00} - \ket{11}}{\sqrt{2}} \\
            \ket{\beta_{10}} & = \frac{\ket{10} + \ket{01}}{\sqrt{2}} \\
            \ket{\beta_{11}} & = \frac{\ket{01} - \ket{10}}{\sqrt{2}}
        \end{align}
        In particular, these are called the Bell States, and form a basis for the two qubit space.
        Note how for $\ket{\beta_{00}}$ a 0-measurement in the first basis i.e. $\ket{0}\bra{0} \otimes I$
        modifies the composite state to $\ket{00}$.
    \end{example}
\end{frame}
\section{Computer Science}
\subsection{Reversible circuits}
\begin{frame}
    \frametitle{Model of Computation}
    In order to leverage Quantum for our computation, we aim to define a model of computation
    that can be used consistently with our postulates. In classical computing some common models are:
    \begin{itemize}
        \item Turing Machine
        \item Circuits
        \item Minsky Register
        \item Petri Nets
    \end{itemize}
    \begin{definition}
        \textbf{Complexity-Theoretic Church-Turing Thesis}
        A probabilistic Turing machine can efficiently simulate any realistic model of computation.
    \end{definition}
    \begin{definition}
        \textbf{Quantum Complexity-Theoretic Church-Turing Thesis}
        A quantum Turing machine can efficiently simulate any realistic model of computation.
    \end{definition}
\end{frame}
\begin{frame}
    \frametitle{Reversible Quantum Circuit}
    The most common model for quantum computation is that of reversible circuits.
    It is motivated by the fact that the laws of Physics are time reversible, and that
    erasing information costs energy.
    \begin{definition}
        \textbf{Landauer's Principle}: Suppose a computer erases a single bit of information.
        The amount of energy dissipated in the environment is at least $k_B T \ln 2$ where
        $T$ is the temperature of the environment, and $k_B$ is Boltzmann's constant
    \end{definition}
\end{frame}
\begin{frame}
    \frametitle{Reversible Circuits}
    Reversible circuits are also a classical computation device. Each reversible gate
    is characterized by the fact that each input configuration uniquely corresponds to an output.
    For example, \texttt{NOT} is reversible as $0 \to 1$ and $1 \to 0$. Instead \texttt{AND} is
    not reversible, as $0$ is the output of $00, 01, 10$.
    We will denote the \texttt{NOT} gate as $X$, as in \ref{Pauli}.\\
    \begin{example}
        Here is an example of how the $X$ gate operates. Since we are operating classically
        so far $\ket{\psi}$ are simply either of $\ket{0},\ket{1}$, but we will see that
        for arbitrary $\ket{\psi} = a \ket{0} +  b \ket{1} $ the quantum $X$ applies the unitary matrix $X$ as expected,
        resulting in $X \ket{\psi} = b \ket{0} +  a \ket{1} $.
        \begin{center}
            \begin{quantikz}
                \lstick{$\ket{\psi}$} & \gate{X} & \rstick{$X \ket{\psi}$} \qw
            \end{quantikz}
        \end{center}
    \end{example}
\end{frame}
\begin{frame}
    \frametitle{Friedkin Gate}
    \begin{example}
        Here is an example of reversible gate, the Toffoli gate.
        \begin{center}
            \begin{quantikz}
                \lstick{$a$} & \ctrl{2} & \rstick{$a$}\qw \\
                \lstick{$b$} & \ctrl{1} & \rstick{$b$}\qw \\
                \lstick{$c$} & \targ{} & \rstick{$c \oplus a b$}\qw \\
            \end{quantikz}
        \end{center}
        It is important as it is universal, as in every classical circuit can be simulated using
        this circuit
        \footnote{Set $c = 1$, then the output of the gate is $\lnot (ab)$. Since \texttt{NAND} is universal, the result follows}.
        In a nutshell, it use $a, b$ as control bits, and if both are set then
        it flips the target bit $c$. It is also easy to see that the $8\times 8$ matrix corresponding
        to the Toffoli is Unitary.
    \end{example}
\end{frame}
\begin{frame}
    \frametitle{Controlled Operations}
    A really interesting kind of operation are the controlled operation. These are operations
    on multiple bit such that the first bit is used as a control. If it is not set nothing happens.
    Else some operation is applied. For any Unitary $U$ we define the controlled-$U$ operation as:
    \begin{center}
        \begin{quantikz}
            & \ctrl{1} & \qw \\
            & \gate{U} \qwbundle[alternate]{}  &  \qwbundle[alternate]{} \\
        \end{quantikz}
    \end{center}
    In matrix form, this corresponds to:
    \begin{align}
        \begin{pmatrix}
            1      & \dots  & 0 \\
            \vdots & \ddots & 0 \\
            0      & \dots  & U
        \end{pmatrix}
    \end{align}
    So the matrix with $1$'s on half of the diagonal, zeros every else and $U$ in the bottom quarter.
\end{frame}
\begin{frame}
    \frametitle{CNOT}
    \begin{example}
        The CNOT gate is just a controlled-$X$ operation, yet it is so important that we have an alternative notation for it.
        \begin{center}
            \begin{quantikz}
                & \ctrl{1} & \qw \\
                & \targ{}  &  \qw \\
            \end{quantikz}
        \end{center}
        In matrix form, it is represented by:
        \begin{align}
            \begin{pmatrix}
                1 & 0 & 0 & 0 \\
                0 & 1 & 0 & 0 \\
                0 & 0 & 0 & 1 \\
                0 & 0 & 1 & 0
            \end{pmatrix}
        \end{align}
        Note how Toffoli is just a CNOT with two control bits, and as such it is sometimes called CCNOT.
    \end{example}
\end{frame}
\subsection{Quantum Gates}
\begin{frame}
    \frametitle{Quantum Gates}
    For now we have just used gates that are classical, in the sense that they transform "bits"
    $\ket{0}, \ket{1}$ into other bits of the same kind. Our aim would be to simulate arbitrary
    Unitary transformations. Turns out that CNOT, and single qubit Unitary operations are all we need
    to compute arbitrary Unitary operations. Furthermore, we can use the Hadamard,
    the phase, the $\pi/8$ \footnote{Next slide} and the CNOT gates to compute any Unitary operation to an arbitrary accuracy. \\
\end{frame}
\begin{frame}
    \frametitle{Quantum Gates}
    \begin{center}
        \text{Hadamard } \qquad
        \begin{quantikz}
            & \gate{H} &  \qw \\
        \end{quantikz} \qquad  $\frac{1}{\sqrt{2}} \begin{pmatrix}
                1 & 1  \\
                1 & -1
            \end{pmatrix}$ \\

        \text{Pauli-}$X$ \qquad
        \begin{quantikz}
            & \gate{X} &  \qw \\
        \end{quantikz} \qquad  $\begin{pmatrix}
                0 & 1 \\
                1 & 0
            \end{pmatrix}$ \\

        \text{Pauli-}$Y$ \qquad
        \begin{quantikz}
            & \gate{Y} &  \qw \\
        \end{quantikz} \qquad  $\begin{pmatrix}
                0 & -i \\
                i & 0
            \end{pmatrix}$ \\

        \text{Pauli-}$Z$ \qquad
        \begin{quantikz}
            & \gate{Z} &  \qw \\
        \end{quantikz} \qquad  $\begin{pmatrix}
                1 & 0  \\
                0 & -1
            \end{pmatrix}$ \\
        \text{Phase} \qquad
        \begin{quantikz}
            & \gate{S} &  \qw \\
        \end{quantikz} \qquad  $\begin{pmatrix}
                1 & 0 \\
                0 & i
            \end{pmatrix}$ \\
        $\pi/8$ \qquad
        \begin{quantikz}
            & \gate{T} &  \qw \\
        \end{quantikz} \qquad  $\begin{pmatrix}
                1 & 0          \\
                0 & e^{i\pi/4}
            \end{pmatrix}$ \\
    \end{center}
\end{frame}
\subsection{Algorithms}
\begin{frame}
    \frametitle{Quantum Speedups}
    We are interested in finding ways to solve some problems more easily then
    with classical methods. As we know, here we have some problems that we think
    are solvable more efficiently on quantum computers (with caveats).
    \begin{itemize}
        \item Deutsch–Jozsa algorithm $2^{n-1}$ calls  to $1$
        \item Quantum Fourier Transform $O(n2^n) \to O(n^2)$
        \item Shor's algorithm $O(\exp(1.9 (\log N)^{1/3})(\log \log N)^{2/3})\to O((\log N)^3)$
        \item Grover's quantum search, $O(n) \to O(\sqrt{n})$
        \item Superdense coding, send $2$ classical bits using only a single qubit
        \item Hidden subgroup problems
    \end{itemize}
    We show quantum circuits for the first two. Also, we stress that
    we have no efficient Quantum algorithm for $\mathcal{NP}$-complete problems.
    As such, we believe (no proof) that $\mathcal{P} \subseteq \mathcal{BPP} \subseteq \mathcal{BQP} \subseteq \mathcal{NP}$ \footnote{
        Actually, a recent paper (https://www.quantum-bits.org/?p=2309) seem to suggest that $\mathcal{BQP} / \mathcal{NP} \neq \emptyset $
    }

\end{frame}
\begin{frame}
    \frametitle{Deutsch-Jozsa}
    Consider a function $f: \{0,1\}^n \to \{0, 1\}$ that is either constant or balanced.
    By balanced we mean that $\sum_s f(s) = 2^{n - 1}$. With how many calls can we determine
    which kind of function is $f$? The classical approach takes at most $2^{n-1} + 1$ applications.
    In the quantum world, we show an algorithm that, given a black box $U_f$ that can compute the Unitary
    $(x, y) \to (x, y \oplus f(x))$, can determine the kind of the function with a single call.
    \begin{center}
        \begin{quantikz}
            \lstick{$\ket{0}$} & \gate{H^{\otimes n}} \qwbundle{n}  & \gate[wires=2]{U_f} & \gate{H^{\otimes n}} & \meter{} \\
            \lstick{$\ket{1}$} & \gate{H}                           & \qw & \qw                  & \qw \\
        \end{quantikz}
    \end{center}

\end{frame}
\begin{frame}
    \frametitle{Deutsch-Jozsa}
    The original input state is $\ket{\psi_0} = \ket{0}^{\otimes n}\ket{1}$.
    This gets transformed to:
    \begin{equation}
        \ket{\psi_1} = \sum_{x \in \{0,1\}^n}\frac{\ket{x}}{\sqrt{2^n}}\left(\frac{\ket{0} -\ket{1}}{\sqrt{2}} \right)
    \end{equation}
    And applying $U_f: \ket{x,y} \to \ket{x, y \oplus f(x)}$ gets us:
    \begin{equation}
        \ket{\psi_2} = \sum_x \frac{(-1)^{f(x)} \ket{x}}{\sqrt{2^n}}\left(\frac{\ket{0} -\ket{1}}{\sqrt{2}} \right)
    \end{equation}
    Finally, evaluating the last Hadamard gates yields:
    \begin{equation}
        \ket{\psi_3} = \sum_z \sum_x \frac{(-1)^{x\cdot z + f(x)}\ket{z}}{2^n}\left(\frac{\ket{0} -\ket{1}}{\sqrt{2}} \right)
    \end{equation}
    Now, the amplitude of $\ket{0}^{\otimes n}$ is $\sum_x (-1)^{f(x)}/2^n$. So if $f$ is constant
    it is $\pm 1$. Otherwise it will be balanced and so equal $0$.
\end{frame}
\begin{frame}
    \frametitle{Quantum Fourier Transform}
    The Discrete Fourier Transform is an operation that has many applications in a variety of problems.
    Classically\footnote{Good visual intro https://www.youtube.com/watch?v=spUNpyF58BY}, it transforms a set of numbers $x_0, \dots, x_{N-1}$ into:
    \begin{equation}
        y_k \equiv \frac{1}{\sqrt{N}} \sum_{j=0}^{N - 1} e^{2\pi i j k / N} x_j
    \end{equation}
    The Quantum Fourier Transform is a linear transformation that,
    for orthonormal basis $\ket{0}, \dots, \ket{N -1}$, computes:
    \begin{equation}
        \ket{j} \to \frac{1}{\sqrt{N}} \sum_{k=0}^{N - 1} e^{2\pi i j k / N} \ket{k}
    \end{equation}
    The connection with the classical is that:
    \begin{equation}
        \sum_{k=0}^{N - 1} x_k \ket{k} \to  \sum_{k=0}^{N - 1} y_k \ket{k}
    \end{equation}
\end{frame}
\begin{frame}
    \frametitle{Quantum Fourier Transform}
    Let us set $N \equiv 2^n$. Then, the best classical algorithm to perform the
    Discrete Fourier Transform uses $\Theta(n2^n) = \Theta(N \log N)$ operations.
    In constrast, we can show a quantum algorithm that uses only $\Theta(n^2)$ gates
    to compute the Quantum Fourier Transform transformation.
    The algorithm uses the fact\footnote{Which is supposed to be elementary smh}
    that the transformation can be written as:
    \begin{equation}
        \ket{j_1, \dots, j_n} \to
        \frac{(\ket{0} + e^{2\pi i 0.j_n}\ket{1}) \dots (\ket{0} + e^{2\pi i 0.j_1 j_2 \dots j_n}\ket{1})}{2^{n/2}}
    \end{equation}
    Where $0.j_l \dots j_m \equiv \sum_{i=l}^{m} \frac{j_i}{2^i}$. \\
    A note is in order: while we have a fast algorithm for the Quantum Fourier Transform, this does
    not give us a fast algorithm for the Discrete version. In order to achieve this speedup we need
    to incorporate the QFT as a routine, which is what Shor's algorithm does in its order finding routine.
\end{frame}
\begin{frame}
    \frametitle{Quantum Fourier Transform}
    Let
    \begin{equation}
        R_k \equiv \begin{pmatrix}
            1 & 0                 \\
            0 & e^{2 \pi i / 2^k}
        \end{pmatrix}
    \end{equation}
    The following circuit computes QFT.
    \begin{center}
        \begin{quantikz}
            \lstick{$\ket{j_1}$} & \gate{H} & \gate{R_2} & \dots  & \gate{R_n} & \qw      & \qw   & \qw            & \qw   & \qw \\
            \lstick{$\ket{j_2}$} & \qw      & \ctrl{-1}  & \dots  & \qw        & \gate{H} & \dots & \gate{R_{n-1}} & \qw   & \qw \\
            & \vdots   &            &        &            &          &       &                &       & \\
            \lstick{$\ket{j_n}$} & \qw      & \qw        & \qw    & \ctrl{-3}  & \qw      & \qw   & \ctrl{-2}      & \dots & \gate{H}\\
        \end{quantikz}
    \end{center}
\end{frame}
\begin{frame}
    \frametitle{QFT}
    \begin{example}
        In the case $N = 4 = 2^2$, we get that $F\ket{j} = \frac{1}{2}\sum_{i = 0}^3 e^{\pi i j k / 2}\ket{k}$
        Alternatively:
        \begin{equation}
            \ket{j_1, j_2} \to \frac{(\ket{0} + e^{\pi i j_2}\ket{1})(\ket{0} + e^{\pi i \left(j_1 + j_2/2\right)}\ket{1})}{2}
        \end{equation}
        The corresponding circuit is:
        \begin{center}
            \begin{quantikz}
                \lstick{$\ket{j_1}$} & \gate{H} & \gate{R_2} & \qw      & \qw & \rstick{$\frac{1}{\sqrt{2}}(\ket{0} + e^{\pi i \left(j_1 + j_2/2\right)}\ket{1})$}   \\
                \lstick{$\ket{j_2}$} & \qw      & \ctrl{-1}  & \gate{H} & \qw & \rstick{$\frac{1}{\sqrt{2}}(\ket{0} + e^{\pi i j_2}\ket{1})$} \\
            \end{quantikz}
        \end{center}
        And swapping the last two completes the operation.
    \end{example}
\end{frame}
\section{Future}
\begin{frame}
    \frametitle{What's next}
    We don't really know!
    \begin{itemize}
        \item Physical realization
        \item Noise
        \item Quantum information
        \item Quantum cryptography
        \item Graph isomorphism?\footnote{Might be solved, paper is 2019 but not published yet afaik https://arxiv.org/abs/1901.06530}
    \end{itemize}
\end{frame}
\begin{frame}
    \frametitle{Learning}
    \begin{itemize}
        \item Linear Algebra: \begin{itemize}
                  \item 3Blue1Brown, Essence of Linear Algebra
                  \item Linear Algebra Done Right. Sheldon Axler
                  \item Linear Algebra and Its Applications, David C. Lay
              \end{itemize}
        \item Physics: \begin{itemize}
                  \item  The Feynman Lectures on Physics
                  \item Minute Physics I guess?
              \end{itemize}
        \item Quantum Computing: \begin{itemize}
                  \item Quantum Computation and Quantum Information, Isaac Chuang and Michael Nielsen
                  \item Brilliant
                  \item Q\# Quantum Programming Language
              \end{itemize}
    \end{itemize}
\end{frame}

\end{document}
